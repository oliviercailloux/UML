\RequirePackage[l2tabu, orthodox]{nag}
\documentclass[french]{beamer}
\input{preamble/packages}
\input{preamble/redac}
\input{preamble/math_basics}
\input{preamble/math_mine}
\input{preamble/draw}

\setbeamertemplate{headline}[singleline]
\setbeamertemplate{footline}[onlypage]

\title{Introduction au cours d’ISI 2}
\subject{UML}
\author{Olivier Cailloux}
\institute[LAMSADE]{LAMSADE, Université Paris-Dauphine}
\date{Version du \today}

\begin{document}
\begin{frame}[plain]
	\tikz[remember picture,overlay]{
		\path (current page.south west) node[anchor=south west, inner sep=0] {
			\includegraphics[height=8mm]{Dauphine-Noir.png}
		};
		\path (current page.south east) node[anchor=south east, inner sep=0] {
			\includegraphics[height=1cm]{LAMSADE95.jpg}
		};
		\path (current page.south) ++ (0, 4em) node[anchor=south, inner sep=0] {
			\scriptsize\textcolor{blue}{\url{https://github.com/oliviercailloux/UML}}
		};
	}
	\titlepage
\end{frame}
\addtocounter{framenumber}{-1}

\section{L’enseignant}
\begin{frame}
	\frametitle{L’enseignant}
	\begin{itemize}
		\item Olivier Cailloux
		\item \hrefblue{mailto:olivier.cailloux@dauphine.fr}{olivier.cailloux@dauphine.fr}
		\item Coordonnées : cf. \hrefblue{https://annuaire.dauphine.psl.eu/annuaire/index.php?param0=fiche&param1=ocailloux}{annuaire} de Dauphine 
		\item Développeur sur projets de recherche
		\item Enseignant chercheur au LAMSADE
	\end{itemize}
\end{frame}

\section{Motivation et objectifs}
\begin{frame}
	\frametitle{Motivation}
	Ingénierie des Systèmes d’Information : 
	\begin{itemize}
		\item Vue haut niveau sur architectures des systèmes 
		\item Vue bas niveau sur la modélisation des besoins
	\end{itemize}
	Ce cours adopte le deuxième point de vue et se concentre sur l’application d’UML
	\begin{itemize}
		\item UML très utile pour développement de haut niveau
		\item Standard largement adopté dans le monde industriel
		\item Longue histoire et encore en évolution
		\item Facilite la communication entre développeurs
		\item Facilite la communication avec les usagers
		\item Démarche de modélisation importante !
	\end{itemize}
\end{frame}

\begin{frame}
	\frametitle{Objectifs}
	\begin{itemize}
		\item Un deuxième cours d’UML ?
		\item UML ? \onslide<2->{Un \emph{langage} de \emph{modélisation}}
		\item Focus différent : le \emph{modèle}
		\item Diagrammes \emph{issus} du modèle, donc \emph{cohérents}
		\item Pour soutenir un projet
	\end{itemize}
	\begin{block}{Qu’apprendra-t-on ?}
		\begin{itemize}
			\item Travailler en équipe sur un projet \emph{avec git}
			\item Utiliser UML pour planifier un développement
			\item Appliquer une méthodologie de développement agile
			\item Distinguer le modèle des diagrammes
			\item Construire et documenter un projet en équipe
		\end{itemize}
	\end{block}
\end{frame}

\section{Organisation}
\begin{frame}
	\frametitle{Organisation du projet}
	\begin{block}{Projet}
		\begin{itemize}
			\item Projets ≠ par équipe (3 à 7), d’envergure
			\item Satisfaction de besoins réels
		\end{itemize}
	\end{block}
	\begin{block}{Itérations, livraisons}
		\begin{itemize}
			\item Développement en \emph{itérations} clôturées par des \emph{livraisons}
			\item Plusieurs livraisons (6 ?) d’un modèle UML évoluant
			\item Livraisons via git
			\item Équipe projet : sous-équipes changeantes à chaque itération
		\end{itemize}
	\end{block}
	\begin{block}{UML VS Java}
		\begin{itemize}
			\item Même équipe et même projet que cours de Java
			\item Sous-équipes non nécessairement identiques
		\end{itemize}
	\end{block}
\end{frame}

\begin{frame}
	\frametitle{Évaluation}
	\begin{itemize}
		\item \hrefblue{https://github.com/oliviercailloux/java-course/blob/main/L3/Projets.adoc\#\%C3\%A9valuation}{Notes} par sous-équipe et par livraison (50\% Projet)
		\item Notes pour exercices notés en séance (50\% CC)
		\item Note finale individuelle (aggrégation des notes)
	\end{itemize}
\end{frame}

\section{Contenu}
\begin{frame}
	\frametitle{Contenu}
	\begin{itemize}
		\item Console et git
		\item Introduction à UML : L, M et U
		\item Eclipse : logiciel de développement multi-usage
		\item Papyrus : Logiciel de modélisation respectueux du standard
		\item Collaboration via GitHub
		\item Diagrammes de cas, de classes, de séquences principalement
		\pgfmathsetmacro{\mycalcNbReleases}{6}
		\pgfmathsetmacro{\mycalcHPerRelease}{7}
		\pgfmathsetmacro{\mycalcHBetweenSessionsMin}{(25 * 4 - 10 * 3 - \mycalcNbReleases * \mycalcHPerRelease) / (10 - 1)}
		\pgfmathsetmacro{\mycalcHBetweenSessionsAvg}{(27 * 4 - 10 * 3 - \mycalcNbReleases * \mycalcHPerRelease) / (10 - 1)}
		\pgfmathsetmacro{\mycalcHBetweenSessionsMax}{(30 * 4 - 10 * 3 - \mycalcNbReleases * \mycalcHPerRelease) / (10 - 1)}
		\item Travail sur projet : \num{\mycalcHPerRelease} heures de travail maison par livraison par personne en moyenne
		\item Travail entre chaque séance : 4 ECTS ⇒ \num[round-mode=places, round-precision=0, mode=text]{\mycalcHBetweenSessionsAvg} heures de travail maison par inter-séance en moyenne
%		\item (Plus précisément : \num[round-mode=places, round-precision=1, mode=text]{\mycalcHBetweenSessionsMin} à \num[round-mode=places, round-precision=1, mode=text]{\mycalcHBetweenSessionsMax})
	\end{itemize}
\end{frame}
\end{document}

\appendix
\makeatletter
\def\insertframenumber{\@roman\c@framenumber}
\def\inserttotalframenumber{\@roman\c@framenumber}
\makeatother

\clearpage\pdfbookmark{License}{License}
\begin{frame}[plain]
	\frametitle{License}
	This presentation, and the associated \LaTeX{} code, are published under the \href{https://opensource.org/licenses/MIT}{MIT license}. Feel free to reuse (parts of) the presentation, under condition that you cite the author.
	
	Credits are to be given to \href{http://www.lamsade.dauphine.fr/~ocailloux/}{Olivier Cailloux}, Université Paris-Dauphine.
\end{frame}
\addtocounter{framenumber}{-1}
\end{document}

\begin{frame}
	\frametitle{Title}
	\begin{block}{Block}
%		\setlength\abovedisplayskip{1 ex}% reduce space above equations
		\begin{itemize}
			\item Item
		\end{itemize}
	\end{block}
	\begin{itemize}
		\item Item
	\end{itemize}
\end{frame}

