\RequirePackage[l2tabu, orthodox]{nag}
\documentclass[french]{beamer}
\input{preamble/packages}
\input{preamble/redac}
\input{preamble/math_basics}
\input{preamble/math_mine}
\input{preamble/draw}

\setbeamertemplate{headline}[singleline]
\setbeamertemplate{footline}[onlypage]

\title{Introduction au cours d’ISI 2}
\subject{UML}
\author{Olivier Cailloux}
\institute[LAMSADE]{LAMSADE, Université Paris-Dauphine}

\begin{document}
\begin{frame}[plain]
	\tikz[remember picture,overlay]{
		\path (current page.south west) node[anchor=south west, inner sep=0] {
			\includegraphics[height=8mm]{Dauphine-Noir.png}
		};
		\path (current page.south east) node[anchor=south east, inner sep=0] {
			\includegraphics[height=1cm]{LAMSADE95.jpg}
		};
		\path (current page.south) ++ (0, 4em) node[anchor=south, inner sep=0] {
			\scriptsize\textcolor{blue}{\url{https://github.com/oliviercailloux/UML}}
		};
	}
	\titlepage
\end{frame}
\addtocounter{framenumber}{-1}

\begin{frame}
	\frametitle{Outline}
	\tableofcontents[hideallsubsections, sectionstyle=shaded/show]
\end{frame}

\section{Motivation et objectifs}
\begin{frame}
	\frametitle{Motivation}
	\begin{itemize}
		\item UML très utile pour développement de haut niveau
		\item Standard largement suivi dans le monde industriel
		\item Longue histoire et encore en évolution
		\item Facilite la communication entre développeurs
		\item Facilite (parfois) la communication avec les clients
		\item Démarche de modélisation importante !
	\end{itemize}
\end{frame}

\begin{frame}
	\frametitle{Objectifs}
	\begin{itemize}
		\item Un deuxième cours d’UML ?
		\item Focus différent !
		\item UML ? \onslide<2->{Un \emph{langage} de \emph{modélisation}}
		\item Diagrammes \emph{issus} du modèle, donc \emph{cohérents}
		\item Pour soutenir un projet
	\end{itemize}
	\begin{block}{Qu’apprendra-t-on ?}
		\begin{itemize}
			\item Travailler en équipe sur un projet
			\item Utiliser UML pour planifier un développement
			\item Appliquer une méthodologie de développement agile
			\item Distinguer le modèle des diagrammes
			\item Construire et documenter un projet en équipe
		\end{itemize}
	\end{block}
\end{frame}

\section{Organisation}
\begin{frame}
	\frametitle{Organisation}
	\begin{block}{Projet}
		\begin{itemize}
			\item Note constituée entièrement par le projet
			\item Projets ≠ par équipe, d’envergure, impl. de besoins réels
			\item Joint au cours de Java
		\end{itemize}
	\end{block}
	\begin{block}{Itérations, livraisons}
		\begin{itemize}
			\item Développement en \emph{itérations} clôturées par des \emph{livraisons}
			\item Plusieurs livraisons (5 ?) d’un modèle UML évoluant
			\item Livraisons via git
			\item Équipe projet : binômes changeants à chaque itération
		\end{itemize}
	\end{block}
%	\begin{block}{Évaluation}
		\begin{itemize}
			\item Notes pour chaque binôme à chaque livraison
			\item Notes pour exercices en séance
			\item Note finale individuelle (aggrégation des notes)
		\end{itemize}
%	\end{block}
\end{frame}

\section{Contenu}
\begin{frame}
	\frametitle{Contenu}
	\begin{itemize}
		\item Console et git
		\item Introduction à UML : L, M et U
		\item Eclipse : logiciel de développement multi-usage
		\item Papyrus : Logiciel de modélisation respectueux du standard
		\item Collaboration avec git et GitHub
		\item Diagrammes de cas, de classes, de séquences principalement
%		\item Travail sur projet : 4 ECTS ⇒ (100 − 10 × 3) / 5 = 14 heures de travail maison par livraison par personne
	\end{itemize}
\end{frame}
\end{document}

\appendix
\makeatletter
\def\insertframenumber{\@roman\c@framenumber}
\def\inserttotalframenumber{\@roman\c@framenumber}
\makeatother

\clearpage\pdfbookmark{License}{License}
\begin{frame}[plain]
	\frametitle{License}
	This presentation, and the associated \LaTeX{} code, are published under the \href{https://opensource.org/licenses/MIT}{MIT license}. Feel free to reuse (parts of) the presentation, under condition that you cite the author.
	
	Credits are to be given to \href{http://www.lamsade.dauphine.fr/~ocailloux/}{Olivier Cailloux}, Université Paris-Dauphine.
\end{frame}
\addtocounter{framenumber}{-1}
\end{document}

\begin{frame}
	\frametitle{Title}
	\begin{block}{Block}
%		\setlength\abovedisplayskip{1 ex}% reduce space above equations
		\begin{itemize}
			\item Item
		\end{itemize}
	\end{block}
	\begin{itemize}
		\item Item
	\end{itemize}
\end{frame}

